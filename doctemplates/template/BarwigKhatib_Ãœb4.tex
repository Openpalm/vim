\documentclass[10pt,a4paper]{article}             % Definiton des Dokuments, Schriftgrösse, Stil
%-----------------------------------------------------------------------------------------------------------------	
\usepackage{uebungsblatt}						  % Einbinden des Übungsblattes, MUSS im gleichen Ordner liegen
%-----------------------------------------------------------------------------------------------------------------	
\usepackage[ngerman]{babel}						  % Deutsche Silbentrennung, Umlaute etc.
\usepackage[utf8]{inputenc}
%-----------------------------------------------------------------------------------------------------------------	
% TikZ Library zum "Zeichnen", bei Bedarf einkommenentieren ! Tex braucht damit glaub länger zum Setzen...
		%\usetikzlibrary{arrows,automata}
		%\usepackage{pgf}
		%\usepackage{tikz}
%-----------------------------------------------------------------------------------------------------------------
\usepackage{stmaryrd} 							  % Blitz Symbol, Widerspruch
\usepackage{amsthm}   							  % Mathepaket
\usepackage{amssymb}  							  % Mathepaket
\usepackage{dsfont}								  % Auch irgendwas für Mathe...
%-----------------------------------------------------------------------------------------------------------------
\usepackage{graphics}						   	  % Einbinden von Grafiken
\usepackage{graphicx}
\usepackage{float}
%-----------------------------------------------------------------------------------------------------------------
\usepackage{fancyvrb}							  % Verbatim Umgebuung Definition
\DefineVerbatimEnvironment{code}{Verbatim}{fontsize=\small}
\DefineVerbatimEnvironment{example}{Verbatim}{fontsize=\small}
\newcommand{\ignore}[1]{}
%-----------------------------------------------------------------------------------------------------------------	
\usepackage{fancyhdr}				              % Package für Kopf und Fusszeile			
\pagestyle{fancy}
%-----------------------------------------------------------------------------------------------------------------	
\renewcommand{\headrulewidth}{0.5pt}			  % Linie der Kopfzeile 
\renewcommand{\footrulewidth}{0.5pt}			  % Linie der Fusszeile
%-----------------------------------------------------------------------------------------------------------------	
\fancyfoot[C]{\thepage}							  % Definitionen der Kopf und Fußzeile
\fancyhead[C]{\thepage}
\fancyhead[L]{Bassel Khatib, Oliver Barwig}
%-----------------------------------------------------------------------------------------------------------------	
\fancypagestyle{ErsteSeite}{                      % Definition für die erste Seite 
\fancyhf{}
\fancyfoot[C]{\thepage}}
%-----------------------------------------------------------------------------------------------------------------	
\begin{document}								  % Beginn des Dokuments
%-----------------------------------------------------------------------------------------------------------------
\thispagestyle{ErsteSeite}						  % Aufruf der Definition für die erste Seite
%-----------------------------------------------------------------------------------------------------------------
%-----------------------------------------------------------------------------------------------------------------
\uebkopfzeile
  {Algorithmen und Datenstrukturen}               % Titel der Veranstaltung
  {Wintersemster 2012, Gruppe D}				  % Semesterangabe, Übungsgruppe
  {Tutorat: Frau Dr. Barbara Pampel}                      			  % Dozenten, Übungsleiter
  {Bassel Khatib, Oliver Barwig}					  % Loesungsblattbearbeiter               
%-----------------------------------------------------------------------------------------------------------------		                  	                                    
\uebtitel
  {Lösungen zum 4.\ Übungsblatt}  			      % Titel (gross und zentriert)
  {Donnerstag, 21.11.2013, 9:00 Uhr }    		                      % Datum der Abgabe
  
  \vspace{8mm}								 	  % Abstand Titel / Erster Abschnitt

% Hier beginnt das richtige Dokument 
%==================================================================================================================

\solution{1}{Quicksort}{6}

	



\newpage


%==================================================================================================================|
% Und hier endet das Dokument wieder... Gute Arbeit :-) 
\end{document}
